\section{Background}
Provide background information and context for your research.

\section{Problem Statement}
State the problem or research question that your dissertation addresses.This is a random citation to demonstrate the bibliography \autocite{karrWholeCellComputationalModel2012}

\section{Research Objectives}
Outline the main objectives and goals of your research.

\section{Thesis Outline}
Provide an overview of the dissertation structure and what each chapter covers.

\subsection{Mathematical Notation Examples}
This subsection shows common math patterns you'll likely need.

Inline maths like $a^2 + b^2 = c^2$ fits within a sentence. Use display math for emphasis:
\[
\int_{0}^{1} x^2\,dx = \tfrac{1}{3}.
\]

Numbered equations can be labelled and referenced using \verb+\eqref{}+:
\begin{equation}
E = m c^2
\label{eq:energy}
\end{equation}
As shown in Einstein's famous \eqref{eq:energy}, energy scales with mass.

Align multistep derivations with alignment points marked by \verb+&+:
\begin{align}
f(x) &= x^2 + 2x + 1 \\
	&= (x+1)^2.
\end{align}

Piecewise definitions and matrices are also available:
\begin{equation}
g(x) = \begin{cases}
 x^2, & x \ge 0, \\
 -x,  & x < 0.
\end{cases}
\end{equation}

\begin{equation}
\mathbf{A} = \begin{bmatrix}
1 & 2 \\
3 & 4
\end{bmatrix}.
\end{equation}

